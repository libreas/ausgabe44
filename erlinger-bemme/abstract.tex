Die Autoren skizzieren, dass insbesondere lokales und regionales Wissen
mit Wikis entsteht und dauerhaft bleibt -- als Regionalia in globalen
offenen Linkzusammenhängen. ``Grass Root Open Access'' bedeutet nicht
nur, Publikationen auf selbst gezimmerte Art und Weise frei und unter
offener Lizenz zu publizieren (``I have published my pdf under a cc
license on my personal website''). ``Grass Root Open Science'' bedeutet
auch, den Inhalt, die Daten und Bilder -- das Wissen einer
publizistischen Arbeit an sich frei, offen und reproduzierbar zu
veröffentlichen. Am Beispiel der ``Wikifizierung'' einer gedruckten,
heimatkundlichen Buchpublikation wird gezeigt, wie mit
Graswurzelstrategien im Wikiversums Open Science entsteht.

Wir skizzieren einen solchen Prozess als `linked open': Methoden und
Effekte regionaler Datenpflege als demokratisierende Praxis mittels
Citizen Science, mit Blick auf Technologien und Gemeinschaften.
Potentiell beeinflussen wir mit offenen, wiki-basierten und damit
dezentralen Wissenssystemen die Kalkulation und Rentabilität
öffentlicher und quasi-öffentlicher Investitionen in Bildungsressourcen,
Informationsinfrastrukturen, Forschung und Entwicklung.

\begin{center}\rule{0.5\linewidth}{0.5pt}\end{center}

The authors outline that local and regional knowledge in particular is
created with wikis and remains permanent. ``Grass Root Open Access''
does not only mean publishing freely and under an open licence in a
self-made way (``I have published my pdf under a cc license on my
personal website''). ``Grass Root Open Science'' means publishing the
content, data and images - the knowledge of the scientific work itself -
freely, openly and reproducibly. Using the example of the
``wikification'' of a printed local history book publication, we show
how grass root strategies in the Wikiverse can be used to create Open
Science. We outline such a process `linked open': Methods and effects of
regional data creation as a democratizing practice with open wiki-based
and thus decentralized knowledge systems, we potentially influence the
calculation and profitability of public and quasi-public investments in
educational resources, in information infrastructures, in research and
development.
