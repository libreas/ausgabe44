Die Autoren skizzieren, dass insbesondere lokales und regionales Wissen
mit Wikis entsteht und dauerhaft bleibt -- als Regionalia in globalen
offenen Linkzusammenhängen. ``Grass Root Open Access'' bedeutet nicht
nur, Publikationen auf selbst gezimmerte Art und Weise frei und unter
offener Lizenz zu publizieren (``I have published my pdf under a cc
license on my personal website''). ``Grass Root Open Science'' bedeutet
auch, den Inhalt, die Daten und Bilder -- das Wissen einer
publizistischen Arbeit an sich frei, offen und reproduzierbar zu
veröffentlichen. Am Beispiel der ``Wikifizierung'' einer gedruckten,
heimatkundlichen Buchpublikation wird gezeigt, wie mit
Graswurzelstrategien im Wikiversums Open Science entsteht.

Wir skizzieren einen solchen Prozess als `linked open': Methoden und
Effekte regionaler Datenpflege als demokratisierende Praxis mittels
Citizen Science, mit Blick auf Technologien und Gemeinschaften.
Potentiell beeinflussen wir mit offenen, wiki-basierten und damit
dezentralen Wissenssystemen die Kalkulation und Rentabilität
öffentlicher und quasi-öffentlicher Investitionen in Bildungsressourcen,
Informationsinfrastrukturen, Forschung und Entwicklung.
