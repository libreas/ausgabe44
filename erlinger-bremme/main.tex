\documentclass[a4paper,
fontsize=11pt,
%headings=small,
oneside,
numbers=noperiodatend,
parskip=half-,
bibliography=totoc,
final
]{scrartcl}

\usepackage[babel]{csquotes}
\usepackage{synttree}
\usepackage{graphicx}
\setkeys{Gin}{width=.4\textwidth} %default pics size

\graphicspath{{./plots/}}
\usepackage[ngerman]{babel}
\usepackage[T1]{fontenc}
%\usepackage{amsmath}
\usepackage[utf8x]{inputenc}
\usepackage [hyphens]{url}
\usepackage{booktabs} 
\usepackage[left=2.4cm,right=2.4cm,top=2.3cm,bottom=2cm,includeheadfoot]{geometry}
\usepackage[labelformat=empty]{caption} % option 'labelformat=empty]' to surpress adding "Abbildung 1:" or "Figure 1" before each caption / use parameter '\captionsetup{labelformat=empty}' instead to change this for just one caption
\usepackage{eurosym}
\usepackage{multirow}
\usepackage[ngerman]{varioref}
\setcapindent{1em}
\renewcommand{\labelitemi}{--}
\usepackage{paralist}
\usepackage{pdfpages}
\usepackage{lscape}
\usepackage{float}
\usepackage{acronym}
\usepackage{eurosym}
\usepackage{longtable,lscape}
\usepackage{mathpazo}
\usepackage[normalem]{ulem} %emphasize weiterhin kursiv
\usepackage[flushmargin,ragged]{footmisc} % left align footnote
\usepackage{ccicons} 
\setcapindent{0pt} % no indentation in captions

%%%% fancy LIBREAS URL color 
\usepackage{xcolor}
\definecolor{libreas}{RGB}{112,0,0}

\usepackage{listings}

\urlstyle{same}  % don't use monospace font for urls

\usepackage{xurl}

\usepackage[fleqn]{amsmath}

%adjust fontsize for part

\usepackage{sectsty}
\partfont{\large}

%Das BibTeX-Zeichen mit \BibTeX setzen:
\def\symbol#1{\char #1\relax}
\def\bsl{{\tt\symbol{'134}}}
\def\BibTeX{{\rm B\kern-.05em{\sc i\kern-.025em b}\kern-.08em
    T\kern-.1667em\lower.7ex\hbox{E}\kern-.125emX}}

\usepackage{fancyhdr}
\fancyhf{}
\pagestyle{fancyplain}
\fancyhead[R]{\thepage}

% make sure bookmarks are created eventough sections are not numbered!
% uncommend if sections are numbered (bookmarks created by default)
\makeatletter
\renewcommand\@seccntformat[1]{}
\makeatother

% typo setup
\clubpenalty = 10000
\widowpenalty = 10000
\displaywidowpenalty = 10000

\usepackage{hyperxmp}
\usepackage[colorlinks, linkcolor=black,citecolor=black, urlcolor=libreas,
breaklinks= true,bookmarks=true,bookmarksopen=true]{hyperref}
\usepackage{breakurl}

%meta
%meta

\fancyhead[L]{Chr. Erlinger \& J. Bemme\\ %author
LIBREAS. Library Ideas, 44 (2023). % journal, issue, volume.
%\href{https://doi.org/10.18452/27071}{\color{black}https://doi.org/10.18452/27071}
{}} % doi 
\fancyhead[R]{\thepage} %page number
\fancyfoot[L] {\ccLogo \ccAttribution\ \href{https://creativecommons.org/licenses/by/4.0/}{\color{black}Creative Commons BY 4.0}}  %licence
\fancyfoot[R] {ISSN: 1860-7950}

\title{\LARGE{Kamptaler Sakrallandschaften im Wikiversum} \vskip 1em \large{Edits mit Versionsgeschichte: Elementarteilchen offener Wissensproduktion am Beispiel eines Citizen Science-Projektes}}% title
\author{Christian Erlinger und Jens Bemme} % author

\setcounter{page}{1}

\hypersetup{%
      pdftitle={Kamptaler Sakrallandschaften im Wikiversum},
     pdfauthor={Christian Erlinger, Jens Bemme},
      pdfcopyright={CC BY 4.0 International},
      pdfsubject={LIBREAS. Library Ideas, 44 (2023).},
      pdfkeywords={Open access, Wikipedia, kollaboratives Schreiben, Sacherschließung},
      pdflicenseurl={https://creativecommons.org/licenses/by/4.0/},
      pdfurl={https://doi.org/},
      pdfdoi={10.18452/},
      pdflang={de},
      pdfmetalang={de}
     }



\date{}
\begin{document}

\maketitle
\thispagestyle{fancyplain} 

%abstracts
\begin{abstract}
\noindent
Die Autoren skizzieren, dass insbesondere lokales und regionales Wissen
mit Wikis entsteht und dauerhaft bleibt -- als Regionalia in globalen
offenen Linkzusammenhängen. ``Grass Root Open Access'' bedeutet nicht
nur, Publikationen auf selbst gezimmerte Art und Weise frei und unter
offener Lizenz zu publizieren (``I have published my pdf under a cc
license on my personal website''). ``Grass Root Open Science'' bedeutet
auch, den Inhalt, die Daten und Bilder -- das Wissen einer
publizistischen Arbeit an sich frei, offen und reproduzierbar zu
veröffentlichen. Am Beispiel der ``Wikifizierung'' einer gedruckten,
heimatkundlichen Buchpublikation wird gezeigt, wie mit
Graswurzelstrategien im Wikiversums Open Science entsteht.

Wir skizzieren einen solchen Prozess als `linked open': Methoden und
Effekte regionaler Datenpflege als demokratisierende Praxis mittels
Citizen Science, mit Blick auf Technologien und Gemeinschaften.
Potentiell beeinflussen wir mit offenen, wiki-basierten und damit
dezentralen Wissenssystemen die Kalkulation und Rentabilität
öffentlicher und quasi-öffentlicher Investitionen in Bildungsressourcen,
Informationsinfrastrukturen, Forschung und Entwicklung.
\end{abstract}

%body
\hypertarget{einleitung}{%
\section{Einleitung}\label{einleitung}}

Das gängige Verständnis von Open Access umfasst den kosten- und
möglichst barrierefreien elektronischen Zugang zu (wissenschaftlicher)
Literatur (OA 2002) oder weiter gefasst die Zugänglichkeit zu Wissen (OA
2003). Der 2003 ausgerufene Übergang zum \enquote{Open-Access-Pa\-ra\-dig\-ma
für elektronische Publikationen} (OA 2003) ist heute von der
Realisierung weit entfernt und maximal eines von Bibliotheken und
Forschungsförderung getriebenes Tätigkeitsfeld. Der Blick in das
Programm der Open-Access-Tage 2023 genügt, um zu sehen, wie bürokratisch
Open Access ist: Kosten-Monitoring, Evaluation und Management der
Transformation -- all das nimmt viel Raum und Zeit ein und kostet
Ressourcen.\footnote{Programm der Open-Access-Tage 2023
  \url{https://open-access-tage.de/open-access-tage-2023-berlin}, Stand
  28.10.2023}

Das Fundament der Bürokratie rund um Open Access und Open Science ist
die wissenschaftliche Produktion. Versuchen wir im Bibliotheksbereich
Publikation, Dokumentation und Vernetzung von wissenschaftlichen
Erkenntnissen -- ganz gleich ob aus institutioneller oder
bürger:in\-nen-wissenschaftlicher Produktion -- losgelöst von
verlegerischen Ansprüchen mit einfachen technischen Hilfsmitteln dem
simplen Grundsatz der kosten- und barrierefreien elektronischen
Zugänglichkeit folgend zu realisieren, so sind die technischen
Hilfsmittel die Gefäße oder Wurzeln, in denen die Forschung wachsen
kann: \enquote{Grassroot Open Science}. Damit begeben wir uns auf den
Boden wissenschaftlicher Kommunikation, wo die publizierten Inhalte von
heute, der Humus sind, auf dem neue Erkenntnisse morgen entstehen
können.

\hypertarget{open-public-humanities-regionalgeschichte-und-offene-wissenschaft}{%
\section{Open Public Humanities -- Regionalgeschichte und Offene
Wissenschaft}\label{open-public-humanities-regionalgeschichte-und-offene-wissenschaft}}

Citizen Science und Open Access wird in der Gemeinsamkeit bislang,
abgesehen von Ausnahmen (Munke 2019), kaum diskutiert. Das mag
einerseits daran liegen, dass Citizen Science oft als ein partizipativer
Methodenkasten institutionalisierter Wissenschaft missverstanden wird,
deren Ergebnisse dann wieder in das \enquote{klassische}
Publikationssystem einfließen. Oder weil es in kaum einer Bibliothek
Aufgabe ist, Bürger:innen in ihren \enquote{privaten Forschungen}
Publikationsunterstützung zu geben. Doch genau darin liegt ein lohnendes
Tätigkeitsfeld, das helfen kann, auch in Bibliotheken ein Methodenset
aufzubauen, das für jede andere Anwendung einer offenen Wissenschaft
genutzt werden kann.

Regional- und Heimatgeschichte ist eines der typischen Forschungsgebiete
für ein Feld mit hohem Einsatz von Bürger:innen. Als Hypothese lässt
sich formulieren, dass gerade in diesem Feld die gedruckte Publikation,
ob als Monographie in kleinster Auflage, im Selbstverlag oder als
Beitrag in regionalen Blättern, einen hohen Stellenwert genießt. Viele
Arbeiten sind stark \enquote{datengetrieben}, wenn darin beispielsweise
Inventare von regionalhistorisch bedeutenden Personen, Bauwerken oder
Ereignissen umfassend recherchiert und mittels Archivmaterialien,
Bilddokumenten und anderen Quellenmaterialien zusammengestellt werden.
Ziel einer \enquote{Grassroot-Open-Science-Transformation} in diesem
Bereich muss daher nicht die bloße kostenfreie, elektronische
Publikation sein, sondern vielmehr das offene Verfügbarmachen der
grundlegenden Inhalte und Daten im Sinne von \enquote{open public
humanities} (Erlinger 2022b). Ein idealer Ort, um heterogene Daten in
strukturierter Form, neben Bildern und Textmaterialien dauerhaft und
offen online zugänglich sowie auch nachnutzbar zu machen, ist das
\emph{Wikiversum} (Kloppenburg \& Schwarzkopf 2016).

\hypertarget{vom-buch-zum-datensatz-im-wikiversum}{%
\section{\texorpdfstring{Vom Buch zum Datensatz im
\emph{Wikiversum}}{Vom Buch zum Datensatz im Wikiversum}}\label{vom-buch-zum-datensatz-im-wikiversum}}

Im Herbst 2020 wurde auf der Website des \enquote{Zeitbrücke-Museums} in
Gars am Kamp (Niederösterreich)\footnote{Website des Zeitbrücke Museums
  Gars am Kamp (Niederösterreich) \url{https://www.zeitbruecke.at/},
  Stand: 28.10.2023} der Hinweis veröffentlicht, dass der damalige
Leiter des Museums, der Maler und Lehrer Anton Ehrenberger,
beabsichtigt, eine umfassende Dokumentation aller sakralen
Kleindenkmäler in der Umgebung des Ortes zu verfassen und in Form eines
Bildbandes im Eigenverlag zu veröffentlichen (Ehrenberger 2022).
Kleindenkmalforschung ist ein Paradebeispiel von datengestützter,
regionalwissenschaftlicher Forschungsarbeit mit einem hohen Grad quasi
vollständiger Dokumentation eines Kulturgutbestandes zu einem bestimmten
Zeitpunkt. Diese Ankündigung war der Anstoß dafür, dass ein Autor dieses
Beitrags den Kleindenkmalforscher kontaktierte, um gemeinsam einen
möglichst umfassenden Open Science-Ansatz auszuprobieren:

\begin{itemize}
\item
  Das (selbst erstellte) Bildmaterial wird unter der Lizenz CC BY 4.0
  auf Wikimedia Commons hochgeladen.
\item
  Jedes beschriebene Objekt erhält ein eigenständiges Wikidata-Item und
  wird mit den im Buch veröffentlichten Informationen weitestgehend
  beschrieben. Diese Daten sind unter CC 0 lizenziert.
\end{itemize}

Das Ergebnis der Zusammenarbeit ist ein öffentlich verfügbarer
Datensatz, der jederzeit wiederverwendet werden kann und dem durch die
grundlegenden Eigenschaften der Software MediaWiki (die zum Beispiel für
die Wikipedia genutzt wird) wesentliche Elemente einer offenen
Wissenschaft automatisch innewohnen:

\begin{itemize}
\item
  Jeder Edit ist Teil einer Versionsgeschichte. Veränderungen am
  Datensatz sind offen zugänglich und können überprüft werden.
\item
  Fehler können korrigiert werden. Fortlaufende, zukünftige
  Veränderungen am Forschungsobjekt können eingepflegt werden.
\item
  Diskussionsseiten erlauben das Besprechen der eingearbeiteten Inhalte.
  Fehler und Zweifel können diskutiert werden.
\end{itemize}

Die Übertragung der Daten aus dem Buch war eine zeitintensive und in
erster Linie manuelle Tätigkeit. Gleichzeitig war dies eine besondere
Art des gründlichen Korrekturlesens und wurde somit zu einer auch
inhaltlichen Unterstützung in der Forschungsarbeit selbst. Das
Übertragen der schriftlichen Informationen in strukturierte Daten in die
Datenbank Wikidata (mit der unter anderem Daten für die Wikipedia
bereitgestellt werden), förderte Ungereimtheiten zu Tage und half Fehler
zu erkennen.

In der nachfolgenden Tabelle ist beispielhaft angegeben, welche
Eigenschaften in Wikidata für ein Objekt, im vorliegenden Fall das
\enquote{Obenaus-Marterl}\footnote{Obenaus-Marterl, Wikidata-Item
  Q38093425 \url{https://www.wikidata.org/wiki/Q38093425}, Stand:
  28.10.2023} gewählt wurden und welche Bedeutung diese tragen
(Entnommen aus Erlinger 2022a).

\begin{table}
\centering
\small

\begin{tabular}{p{0.2\textwidth}|p{0.2\textwidth}|p{0.5\textwidth}}
\hline
\textbf{Wikidata-}\textbf{Eigenschaft} & \textbf{Beschreibung} & \textbf{Beispiel (}\textbf{Obenaus-Marterl, Wikidata-Item Q38093425)} \\
\hline
Label & Mehrsprachiges \enquote{Label} als textuelle Bezeichnung des Kleindenkmals & \enquote{Obenaus-Marterl} \\
\hline
Description & Mehrsprachige Beschreibung des Datenobjektes & de: \enquote{Bildstock in Gars am Kamp} en: \enquote{wayside shrine, Gars am Kamp (Lower Austria)} \\
\hline
P31 \enquote{ist ein} & Benutzt ein anderes Wikidata-Objekt zur ontologisch-taxonomischen Erfassung & \enquote{wayside shrine} (WD:Q3395121) \\
\hline
P18 \enquote{Bild} & Bild – exemplarisch & Bilddatei (Obenausmarterl 2020.jpg) \\
\hline
P571 \enquote{Erstellung} & Datum der Erstellung & 19. Jahrhundert \\
\hline
P17 \enquote{Staat} & Staat i.d. Item (Denkmal) liegt & Österreich (WD:Q40) \\
\hline
P131 \enquote{liegt in Verwaltungseinheit} & Verlinkt mit der politischen Gemeinde & Gars am Kamp (WD:Q174400) \\
\hline
P276 \enquote{Ort} & Katastralgemeinde & Thunau am Kamp (WD:1366206) \\
\hline
P571 \enquote{Erstellung} & Datum der Erstellung des Denkmals (unterschiedliche Genauigkeitsgrade bis zu Tagesdatum möglich) & 19. Jahrhundert \\
\hline
P5607 \enquote{liegt in kirchlicher Verwaltungseinheit} & Verlinkt mit der Pfarrgemeinde & Pfarre Gars am Kamp (WD:Q105316672) \\
\hline
P625 \enquote{Geo-Koordinaten} & Erfasst das Koordinatendatum & 48°35'58.2"N, 15°38'51.0"E \\
\hline
P1343 \enquote{beschrieben in Quelle} & Erfasst (bibliographische) Quellen in denen das Denkmal beschrieben wurde. & * Kamptaler Sakrallandschaften von Anton Ehrenberger (2022) (WD:Q108867519)

* Österreichische Kunsttopographie (1911) (WD:Q105824260) \\
\hline
P2951 Bundesdenkmalamt ID & In Wikidata sind mehrere tausend externe Datenbanken verknüpft, bspw. BDA & 73254 \\
\hline
P9154 HERIS-ID & Neue Bundesdenkmalamt-Datenbank & 60872 \\
\hline
Sitelink: Wikimedia Commons & Mit \enquote{Sitelink} kann auf andere Seiten des Wiki*Versums verlinkt werden (Wikipedia-Seite etc.) & Category:Obenaus-Marterl (\href{https://commons.wikimedia.org/wiki/Category:Obenaus-Marterl}{https://commons.wikimedia.org/wiki/Category:Obenaus-Marterl}) \\
\hline

\end{tabular}

\end{table}

Die ausgewählten Datenfelder der Tabelle sind exemplarisch jene Werte,
die für das beschriebene Kleindenkmal bekannt sind. Die Flexibilität von
Wikidata erlaubt aber die Erfassung weiterer detaillierter
Informationen, beziehungsweise, die Daten können jederzeit um weitere
\enquote{Datenfelder} ergänzt sowie auf Normdaten verlinkt werden:

\begin{itemize}
\item
  Verlinkung und damit Referenzierung einer Vielzahl externer
  Datenbanken oder Websites wie zum Beispiel marterl.at
  (\url{https://www.marterl.at/}) oder die Gemeinsame Normdatei der
  Deutschen Nationalbibliothek (GND).
\item
  Es können signifikante Ereignisse (Renovierung, Weihe) hinzugefügt und
  detailliert werden.
\item
  Das Denkmal kann noch näher beschrieben werden: Baustil oder Epoche,
  Größe, verwendetes Material et cetera.
\item
  Erfassung von Details wie Abbildungen oder Inschriften.
\end{itemize}

Gerade die Verlinkungen der beschriebenen Objekte mit gegebenenfalls der
GND, mit der Datenbank des österreichischen Bundesdenkmalamtes oder
anderen Quellen, war eine Tätigkeit, die klassisch bibliothekarisch
motiviert war und die in der Buchpublikation befindlichen Inhalte
darüber hinaus inhaltlich kontextualisierte. Umgekehrt wurden im
gedruckten Buch sehr viele historische Quellenangaben zu den einzelnen
Dokumenten notiert -- bei vielen dieser Quellen handelt es sich
ebenfalls um graue Literatur. Abgesehen von der österreichischen
Kunsttopographie, die Open Access verfügbar ist,\footnote{Digitalisat
  der Österreichischen Kunsttopographie (1889), TUGraz DIGITAL Library,
  \url{https://diglib.tugraz.at/oesterreichische-kunsttopographie-1889},
  Stand: 28.10.2023} wurden noch keine weiteren Quellenangaben für das
vorgestellte Projekt in Wikidata systematisch erfasst.

Die Erfassung dieser Daten im \emph{Wikiversum} lädt unmittelbar dazu
ein, solche Informationen mit anderen Daten und Objekten zu verbinden,
die im Buch selbst nicht beschrieben sind, zum Beispiel aus
Platzgründen. Dazu zählen neben der erwähnten Normdaten-Verlinkung auch
Verbindungen zu Wikipedia-Artikeln oder die Vernetzung mit
Bildmaterialien in Wikimedia Commons (und dazu die Erzeugung von
Kategorien in Commons zur Sammlung von mehreren Mediendateien des
gleichen Denkmals).

Da die Daten nicht nur frei verfügbar und nutzbar, sondern auch
beständig weiter editierbar sind - \enquote{it's a wiki!} - wurde
bereits während der Drucklegung des Buches die Forschung fortgeführt. So
konnten Renovierungen von Denkmälern erfasst werden oder auch Fehler,
die trotz mehrfacher Korrekturläufe erst spät entdeckt wurden, zumindest
im offenen Datensatz behoben werden. Dem Datenmodell von Wikidata sei
Dank, brauchen solche Fehler nicht einfach überschrieben werden, sondern
können als \enquote{deprecated} mit der entsprechenden Quellenangabe
erfasst werden. Dies erlaubt ein digitales Erratum zur Publikation zu
erstellen.\footnote{Wikidata SPARQL-Query \enquote{Errata der
  Publikation \enquote*{Kamptaler Sakrallandschaften}},
  \url{https://w.wiki/7vrL)}, Stand: 28.10.2023. (Dank an Lucas
  Werkmeister (WMDE) für den Support bei dieser Abfrage!)}

Das Weiterführen des Datensatzes durch Veränderungen der Denkmäler
(Renovierungen, Verschwinden, Neuerrichtung et cetera), durch
Präzisierung von Angaben, die in der Publikation nicht erfasst wurden
(beispielsweise physische Abmessungen), Vernetzung mit anderen
Datenquellen -- all das ist kontinuierlich möglich. Dazu soll die
Veröffentlichung im \emph{Wikiversum} anregen: Jede:r kann editieren,
auch ohne Registrierung, und helfen diesen Datenbestand weiter zu
entwickeln.

Die etwaige Nachnutzung der Daten macht dieses
bürger:innen-wissenschaftliche Forschungsvorhaben aber erst richtig
lebendig:

\begin{itemize}
\item
  Neue Forschungsarbeiten zu diesem Themengebiet können nahtlos an das
  Inventar aus dem Jahr 2022 anschließen und Vergleichsarbeiten
  aufbauen.
\item
  Bilder können dank der offenen Lizenz für jeden Zweck verwendet
  werden.
\item
  Online-Dokumentation: Mit einer Website\footnote{Online-Dokumentation
    \enquote{Kamptaler Sakrallandschaften},
    \url{https://kamptalersakrallandschaften.gitlab.io/}, Stand
    28.10.2023} wurde gezeigt, wie ohne großen Aufwand die vorhandenen
  Daten der Publikation online präsentiert werden können, und
  gleichzeitig, welche Analysemöglichkeiten mit den einfachen
  Visualisierungen des Wikidata-SPARQL-Endpoints durchführbar sind.
\end{itemize}

\hypertarget{neue-akteurinnen-fuxfcr-die-regionale-offene-wissensproduktion}{%
\section{Neue Akteur:innen für die regionale offene
Wissensproduktion}\label{neue-akteurinnen-fuxfcr-die-regionale-offene-wissensproduktion}}

Das hier dokumentierte Projekt verknüpft traditionsreiche Felder der
Heimatforschung oder des Denkmalschutzes und der Landeskunde mit Open
Data-Werkzeugen des \emph{Wikiversums} und Arbeitsweisen dort.

Welche Bedeutung aber haben Bibliotheken und insbesondere Landes- oder
Kantonsbibliotheken für diese Art \enquote*{Grassroot Open Access}?
Welche Rollen können Bibliotheken spielen, beziehungsweise die Menschen,
die in Bibliotheken arbeiten, um solche Kooperationen
\enquote*{regionaler Wissensproduktion} zu begleiten? Im idealen Fall
kollaborativ nicht erst am Ende eines Forschungsprozesses als
verlängerte Werkbank für gedruckte oder PDF-Publikationen, sondern als
Co-Kreation einer nutzbaren und editierbaren Datenveröffentlichung mit
Versionsgeschichte. Im Idealfall auch nicht nur im Ehrenamt wie im
vorliegenden Fall, sondern als Bestandteil professioneller Unterstützung
und etwaiger Bibliotheksdienste.

Bei der Erschliessung der \enquote{Kamptaler Sakrallandschaften} hat
sich erst im Verlauf gezeigt, dass mehr Aufwand entsteht als gedacht.
Die Daten waren nicht, wie erhofft, bereits tabellarisch erfasst,
sondern nur als Fließtext vorhanden. Der Text musste also Satz für Satz
gelesen werden, um die wichtigsten Informationen für eine strukturierte
Erfassung zu ermitteln. Andere Informationen wie geographische
Koordinaten oder die Zuteilung zu den kleinsten Gebietskörperschaften
wurden erst eigenständig ermittelt. Von der anfänglichen Vorstellung
bloß \enquote{Datenteiler} zu sein, stand man auf einmal mitten im
Forschungsprozess selbst.

Werden regionale \enquote{Open GLAM Labore} zukünftig in der Lage sein,
eine offene Datenkultur auch mit Regionalbibliotheken für regionale
Forschungsprozesse, deren Edition und Publikation zu verstärken, zum
Beispiel in der Heimatforschung? Seit circa 2018 formierte sich eine
Gemeinschaft international orientierter Bibliotheken, um eigene
Innovationslabore (Open Glam Labs) für (eigene) offene Datenkulturen zu
gründen und zu vernetzen (OA 2018).

Regionale \enquote{Open GLAM Labore} für \enquote{Graswurzel-Open Access
für Regionalia} wären ein nächster Schritt, um diese Methoden auch
\enquote{vor Ort} als mit Regionalbezug zu teilen -- sei es durch
Landesbibliotheken, mit Fachstellen der öffentlichen Bibliotheken, mit
Staats- und Stadtarchiven, mit Hackathon- und Editathon-Initiativen,
beim nächsten Wikipedia-Stammtisch oder mit lokalen Open
Data-Gemeinschaften.

Ergo: Schafft Open GLAM-Labore für viele (Ebenen)!

Edits mit Versionsgeschichte sind, hier im skizzierten Projekt, im
\emph{Wikiversum} und für verknüpfte Datenbestände die Elementarteilchen
der Wissensproduktion mit Wikidata und Wikimedia Commons -- in den
offenen Forschungsdaten, deren crowdorientierter Datenedition,
Publikation und Erschließung auf der Meta(daten)ebene. Die kleinste
messbare Größe solcher offenen Publikationsprozesse ist \enquote*{1 Edit
(+/- 0 Byte)}. Wikidata und andere offene Wissensplattformen sind in
diesem Sinne Gefäße für regionales Wissen. Offene Texte, offene Daten
und offene Metadaten können von jeder und jedem selbst geschaffen und
veredelt werden -- angereichert, verknüpft, benutzt, gespeichert und neu
beschrieben. Edits mit Versionsgeschichte sind dann eine Einheit zur
Messung und der Maßstab der Openness für Regionalia.

\hypertarget{bibliographie}{%
\section{Bibliographie}\label{bibliographie}}

Ehrenberger, Anton 2022. \emph{Kamptaler Sakrallandschaften: Gars -
Schönberg - St.~Leonhard: mit den sieben Pfarren des Pfarrverbandes
Gars: Gars, Freischling, Plank, Schönberg, Stiefern, St.~Leonhard,
Tautendorf}. Zeitbrücke-Museum.

Erlinger, Christian 2022a. Kamptaler Sakrallandschaften - Linked Open
Data. \url{https://osf.io/7hvem/} {[}Stand 2022-10-02{]}

Erlinger, Christian 2022b. Open Public Humanities.
\url{https://zenodo.org/doi/10.5281/zenodo.6759428} {[}Stand
2023-10-28{]}.

Kloppenburg, Julia \& Schwarzkopf, Christopher 2016. Citizen Science im
Wikiversum. In K. Oswald \& R. Smolarski, hg. \emph{Bürger Künste
Wissenschaft: Citizen Science in Kultur und Geisteswissenschaften}.
91--102. \url{https://doi.org/10.22032/dbt.39058}

Munke, Martin 2019. Landesbibliographie und Citizen Science.
Kooperationsmöglichkeiten für Bibliotheken und Wiki-Communities am
Beispiel der Sächsischen Bibliografie. \emph{Regionalbibliographien:
Forschungsdaten und Quellen des kulturellen Gedächtnisses: Liber
amicorum für Ludger Syré}13.

OA 2003. Berliner Erklärung über den offenen Zugang zu
wissenschaftlichem Wissen.
\url{https://openaccess.mpg.de/68053/Berliner_Erklaerung_dt_Version_07-2006.pdf}.

OA 2002. \emph{Budapest Open Access Initiative (German Translation)}.
\url{https://www.budapestopenaccessinitiative.org/read/german-translation/}
{[}Stand 2023-10-28{]}.

OA 2018. \emph{Open a Glam Lab}. \url{https://glamlabs.pubpub.org/} {[}Stand
2020-08-27{]}.

%autor
\begin{center}\rule{0.5\linewidth}{0.5pt}\end{center}

\textbf{Christian Erlinger} hat Raumplanung und Politikwissenschaft
studiert und ist seit 2013 im Bibliotheksbereich tätig. Aktuell ist er
Mitarbeiter an der ZHB Luzern (CH) und betreut ZentralGut.ch das
digitale Kulturgutportal der Zentralschweiz. Er ist Wikidata-Enthusiast
(\#DieDatenlaube) und setzt sich für das verstärkte Zusammenspiel von
GLAM-Institutionen und dem Wikiversum ein. \textbf{ORCID}:
0000-0001-7872-9617 \textbar{} \textbf{Mastodon}:
librerli@openbiblio.social

\textbf{Jens Bemme} studierte Verkehrswirtschaft und
Lateinamerikastudien. Heute interessiert er sich für Dorfbacköfen und
historisches Radfahrerwissen um 1900 in der Oberlausitz und der
Ostseeprovinzen. Mit der `Datenlaube' und Christian Erlinger erschließt
er Wikisource-Volltexte der Illustrierten `Die Gartenlaube' offen in
Wikidata. Als Mitarbeiter der SLUB Dresden begleitet Jens
landeskundliche Citizen Science-Initiativen insbesondere mit den
digitalen Werkzeugen und Gemeinschaften der Wikimedia-Bewegung.
\textbf{ORCID}: 0000-0001-6860-0924 \textbar{} \textbf{Mastodon}:
JensB@openbiblio.social

\end{document}