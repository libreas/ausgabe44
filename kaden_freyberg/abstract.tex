In den 2010er Jahren begann die Diskussion über Makerspaces in
Bibliotheken als Reaktion auf die Digitalisierung und den Wunsch nach
neuen Nutzungsmöglichkeiten. Diese Räume erweiterten Bibliotheken zu
kollaborativen Zentren für digitales Gestalten und gemeinsame
Tool-Anwendung. Sie fördern dabei kreative Zusammenarbeit und
Kompetenzvermittlung in Hinblick auf digitale Verfahren. Der vorliegende
Text ist als Vorlage für die weitere Beschäftigung mit Makerspaces und
Library Labs in wissenschaftlichen Bibliotheken gedacht. Dafür
betrachtet er Beispiele aus wissenschaftlichen Bibliotheken und
diskutiert Ziele, Konzepte sowie Ausstattungsanforderungen. Diese werden
in einer Übersicht zu Konzeptmerkmalen sowie einer Makerspace-Matrix
zusammengefasst. Es wird deutlich, dass Digital Makerspaces und Library
Labs agile Innovationsorte sind, die über unterschiedliche Formate und
Methoden wie Co-Creating und Peer-to-Peer-Lernen interdisziplinäre
Zusammenarbeit und inklusives Lernen fördern.

\begin{center}\rule{0.5\linewidth}{0.5pt}\end{center}

In the 2010s, the discussion about makerspaces in libraries began as a
reaction to digitalisation and the desire for new ways of use. These
spaces expanded libraries into collaborative centres for digital
creation and joint tool use. They encourage creative collaboration and
skills transfer with regard to digital processes. This text is intended
as a starting point for further exploration of makerspaces and library
labs in academic libraries. To this end, it examines examples from
academic libraries and discusses objectives, concepts as well as
equipment requirements. These are summarised in an overview of concept
features and a makerspace matrix. It emerges that digital makerspaces
and library labs are agile places of innovation that promote
interdisciplinary collaboration and inclusive learning through different
formats and methods such as co-creation and peer-to-peer learning.
