\textbf{Kurzfassung}: Der vorliegende Beitrag beleuchtet den Ansatz des
scholar-led publishing und zeigt auf, welche Zusammenhänge zwischen
scholar-led Initiativen und der \enquote*{klassischen} Open Access-Bewegung
bestehen. Nach einer kurzen Diskurseinordnung leitet der Beitrag
diachron ab, wie scholar-led Initiativen aus den Geistes- und
Sozialwissenschaften schon früh und parallel zu den weithin rezipierten
Entwicklungen aus dem medizinisch-naturwissenschaftlichen Bereich der
1990er Jahre auf eigene Weise wichtige Impulse zur Öffnung von
Publikationskulturen setzten. Dazu werden exemplarisch eine Vielzahl von
medien- und kulturwissenschaftlichen scholar-led Journals, Buchverlagen
sowie weiterreichenden Netzwerk- und Infrastruktur-Initiativen entlang
der größeren Entwicklungen der letzten vier Jahrzehnte hin zur
Digitalisierung von wissenschaftlichen Publikationspraktiken und dem
sich daraus ergebenden Potential eines alternativen Publikationssystems,
welches gemeinschaftliche Kollaboration unter den Vorzeichen von
Gemeinnützigkeit über kommerzorientierten Wettbewerb stellt.
